\poem{Laughter}{Laughter = \frac{H \cdot E^{1/2}}{1 + \exp(-F)} + S}{\item $H$: \index{Humor}\textit{Humor}. The innate appeal of the joke or funny situation. Varied by personal taste, culture, and the context of the joke, high-level humor is more likely to induce laughter.
\item $E$: \index{Mood}\textit{Mood}. Prior feelings ranging from joy to stress. A better mood primes individuals for stronger laughter responses to humor.
\item $F$: \index{Familiarity}\textit{Familiarity}. How well the individual knows the style or type of humor. Some familiarity is beneficial for laughing at a joke; too much or too little can lessen this effect.
\item $S$: \index{Social Influence}\textit{Social Influence}. Reflects the effect of being among others on laughter. People laugh more and louder in groups due to shared understanding and laughter's contagious nature.
}