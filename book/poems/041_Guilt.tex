\poem{Guilt}{Guilt = \frac{I}{A + R} \cdot (E - C) + M}{\item $I$: \index{Intentionality}\textit{Intentionality}. Intentionality of the act. Reflects how much the action was done on purpose. A higher value indicates a more intentional act, which typically increases feelings of guilt.
\item $A$: \index{Apologies}\textit{Apologies}. Apologies and attempts to make amends. Sum of sincere efforts to apologize or compensate for the hurtful action. A higher count can significantly reduce guilt levels.
\item $R$: \index{Remorse}\textit{Remorse}. Remorse felt by the individual. It quantifies the regret and emotional distress over the action, which can help in reducing guilt when sincere remorse is shown.
\item $E$: \index{Expectations}\textit{Expectations}. Expectations from others or oneself that were violated. This variable quantifies how much an action deviated from expected norms or values, which often amplifies guilt.
\item $C$: \index{Circumstances}\textit{Circumstances}. Circumstances outside of one's control that influenced the action. These can include factors like pressure from others or unforeseen events, mitigating the impact of the action on guilt levels.
\item $M$: \index{Mindfulness}\textit{Mindfulness}. Mindfulness and self-awareness about the situation. Reflects the level of understanding and contemplation over one's actions and their impact. Higher mindfulness can slightly offset guilt, fostering self-compassion.
}