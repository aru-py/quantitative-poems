\poem{Pain}{Pain = \frac{T \cdot (S + E)}{R + C}}{\item $T$: \index{Threshold}\textit{Threshold}. The pain threshold is how much discomfort a person can endure. It's shaped by both mental and physical aspects, affecting when pain becomes unbearable.
\item $S$: \index{Severity}\textit{Severity}. Severity denotes the intensity level of the discomfort-causing factor, such as injury or illness.
\item $E$: \index{Emotion}\textit{Emotion}. Emotion, or the mental state of an individual, can amplify pain. A distressed mental state often heightens pain perception.
\item $R$: \index{Resilience}\textit{Resilience}. Resilience reflects a person's ability to withstand or recover from discomfort. High resilience can diminish pain's effect.
\item $C$: \index{Comfort}\textit{Comfort}. Comfort involves external factors like medication or support that can lessen pain.
}