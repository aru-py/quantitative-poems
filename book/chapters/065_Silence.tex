\chapter{Silence}

\begin{equation}
\text{Silence} = \frac{P}{c} \cdot e^{-\frac{t}{T}} - \alpha \cdot (M + E) + L
\end{equation}

\textbf{Where:}

\begin{itemize}
    \item $P$: The initial power or intensity of a preceding sound or noise.
    \item $c$: The constant representing spatial and environmental factors that affect sound propagation.
    \item $t$: Time elapsed since the initial sound or noise.
    \item $T$: The temporal threshold for human perception of sound continuity, reflecting how quick sounds are considered separate.
    \item $\alpha$: The coefficient measuring the impact of human or environmental activity in disrupting silence.
    \item $M$: The intrinsic mental noise or internal dialogue within an individual.
    \item $E$: External disturbances or ambient noise level.
    \item $L$: The baseline silence level, representing an ideal state of complete silence or the minimum perceivable sound level for a human.
\end{itemize}