\chapter{Anger}

\begin{equation}
\text{Anger} = \frac{(I + E) \cdot (F - C) + D}{T + R}
\end{equation}

\textbf{Where:}

\begin{itemize}
    \item $I$: Injustice perceived by the individual.
    \item $E$: Emotional sensitivity of the individual.
    \item $F$: Frustration encountered in daily activities.
    \item $C$: Capacity of the individual to cope with adversity.
    \item $D$: Influence of past traumas or disappointments.
    \item $T$: Threshold of tolerance for the individual.
    \item $R$: Resources available (emotional, social support) to mitigate anger.
\end{itemize}