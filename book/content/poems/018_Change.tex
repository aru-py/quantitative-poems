\poem{Change}{Change = \frac{D \cdot M \cdot T^2}{R + I}}{\item $D$: \index{Desire}\textit{Desire}. The internal motivation and yearning for something different, acting as the primary driving force that initiates all meaningful transformation in human experience.
\item $M$: \index{Momentum}\textit{Momentum}. The sustained energy and consistent action taken toward transformation, building upon itself like a snowball effect that accelerates personal evolution.
\item $T$: \index{Time}\textit{Time}. The duration and patience required for change to manifest, squared to show how extended periods exponentially amplify the depth of transformation achieved.
\item $R$: \index{Resistance}\textit{Resistance}. The internal and external forces that oppose change, including fear, comfort zones, societal expectations, and the natural human tendency toward stability.
\item $I$: \index{Inertia}\textit{Inertia}. The psychological tendency to remain in current patterns and habits, representing the gravitational pull of familiar routines that must be overcome for growth.}{This equation reveals change as a delicate dance between catalytic forces and stabilizing resistances. Desire ignites the spark, momentum sustains the journey, and time's square demonstrates how patience compounds transformation exponentially. Yet resistance and inertia form the denominator - not as enemies, but as necessary counterweights that ensure change is meaningful rather than chaotic. Like a river carving through stone, true change requires persistent force applied over time, gradually wearing down the barriers that once seemed insurmountable.}