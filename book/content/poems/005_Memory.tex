\poem{Memory}{Memory = \frac{E \cdot R \cdot T^2}{F + A}}{\item $E$: \index{Emotion}\textit{Emotion}. The intensity of feelings associated with an experience, which significantly enhances memory formation and recall through neurochemical reinforcement.
\item $R$: \index{Repetition}\textit{Repetition}. The frequency of exposure to information or experiences, strengthening neural pathways through practice and reinforcing long-term retention.
\item $T$: \index{Time}\textit{Time}. The duration since encoding, where memories strengthen through consolidation but may also fade without reinforcement, creating a complex temporal relationship.
\item $F$: \index{Forgetting}\textit{Forgetting}. The natural decay of unused memories and interference from competing information, representing the brain's selective filtering of experiences.
\item $A$: \index{Age}\textit{Age}. The biological factor affecting memory formation and retrieval, where cognitive changes over time influence both capacity and accessibility of stored information.}{This equation reveals memory as an intricate dance between preservation and loss. Emotional intensity and repetition work together to etch experiences deeper into our minds, while time serves as both ally and adversary - consolidating precious moments yet allowing others to fade. The denominators of forgetting and age remind us that memory is not a perfect recording, but a living, breathing process that shapes who we are through what we choose to remember and release.}