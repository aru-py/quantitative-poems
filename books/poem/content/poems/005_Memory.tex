\poem{Memory}{Memory = \frac{S \cdot E^{\alpha}}{1 + \lambda t} \cdot e^{-\beta \Delta t}}{\item $S$: \index{Significance}\textit{Significance}. The emotional and personal importance of an experience when first encoded, determining the initial memory strength and likelihood of long-term retention.
\item $E$: \index{Emotion}\textit{Emotion}. The intensity of emotional arousal during memory formation, which acts as a powerful amplifier for encoding strength and retrieval accessibility.
\item $\lambda$: \index{Interference}\textit{Interference}. The rate at which new experiences and competing memories interfere with and gradually weaken the accessibility of stored memories over time.
\item $\Delta t$: \index{Duration}\textit{Duration}. The elapsed time since the memory was first formed, representing the natural decay process that affects all memories through biological forgetting mechanisms.}{This equation reveals memory as a delicate dance between preservation and decay. Significance and emotion work together exponentially to forge strong initial impressions, while time's relentless passage creates competing forces of interference and natural forgetting. The mathematical structure shows how our most meaningful and emotionally charged experiences resist time's erosion, yet even the most profound memories must contend with the inevitable fading that makes room for new experiences in the theater of consciousness.}