\poem{Envy}{Envy = \frac{P \cdot \ln(D + 1)}{S^2 \cdot G}}{\item $P$: \index{Perception}\textit{Perception}. Our subjective interpretation of others' advantages, often distorted by incomplete information and social comparison, amplifying what others seem to possess.
\item $D$: \index{Disparity}\textit{Disparity}. The perceived gap between what others have and what we possess, whether material wealth, relationships, achievements, or opportunities in life.
\item $S$: \index{Security}\textit{Security}. Our internal sense of self-worth and confidence in our own path, which when strong, acts as a powerful shield against envious thoughts and comparisons.
\item $G$: \index{Gratitude}\textit{Gratitude}. The practice of appreciating what we already possess, serving as a natural antidote to envy by shifting focus from lack to abundance in our lives.}{This equation reveals envy as perception multiplied by the logarithm of disparity, divided by the square of security and gratitude. The logarithmic relationship shows that envy grows rapidly at first but plateaus as disparities increase. Security's squared effect demonstrates how self-confidence powerfully diminishes envy, while gratitude acts as a constant divisor, reducing envy's intensity through appreciation of our own blessings.}