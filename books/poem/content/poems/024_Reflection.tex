\poem{Reflection}{Reflection = \frac{W \cdot \ln(E + 1) \cdot \sin(\theta)}{T^2 + B}}{\item $W$: \index{Wisdom}\textit{Wisdom}. The accumulated understanding and insight gained through life experiences, serving as the foundation that amplifies the capacity for meaningful self-examination and growth.
\item $E$: \index{Experience}\textit{Experience}. The breadth of lived moments and encounters that provide the raw material for reflection, where each experience adds logarithmic depth to contemplative understanding.
\item $\theta$: \index{Perspective}\textit{Perspective}. The angle or viewpoint from which one examines their life, representing the oscillating nature of how we view ourselves and our circumstances over time.
\item $T$: \index{Time}\textit{Time}. The temporal distance from events being reflected upon, which paradoxically can both clarify and obscure understanding as it increases exponentially in its effects.
\item $B$: \index{Bias}\textit{Bias}. The cognitive prejudices and preconceptions that cloud objective self-examination, acting as a constant barrier that diminishes the purity of reflective insight.}{This equation reveals reflection as wisdom illuminating experience through the lens of perspective, tempered by time's passage and our inherent biases. The logarithmic relationship with experience shows how each new encounter adds diminishing but meaningful depth to our capacity for self-understanding. The sinusoidal perspective captures how our viewpoint oscillates like light through a prism, sometimes revealing brilliant insights, other times casting shadows. Time squared in the denominator demonstrates how distance from events can exponentially complicate our ability to see clearly, while bias remains a persistent fog that dims the mirror of self-knowledge.}