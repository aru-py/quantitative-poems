\poem{Jealousy}{Jealousy = \frac{P \cdot I^2 \cdot \ln(C + 1)}{S \cdot T}}{\item $P$: \index{Possessiveness}\textit{Possessiveness}. The degree of attachment and desire to control or own something or someone, representing the fundamental drive that fuels jealous feelings and behaviors.
\item $I$: \index{Insecurity}\textit{Insecurity}. Personal feelings of inadequacy and self-doubt that amplify jealous responses, squared to show its exponential impact on emotional volatility and perception.
\item $C$: \index{Comparison}\textit{Comparison}. The mental process of measuring oneself against others, logarithmically scaled as comparisons compound gradually but persistently in consciousness.
\item $S$: \index{Security}\textit{Security}. Inner confidence and trust in relationships that acts as a stabilizing force, reducing jealous tendencies through emotional grounding and self-assurance.
\item $T$: \index{Trust}\textit{Trust}. Faith in others' loyalty and intentions that serves as a protective denominator, diminishing jealousy by fostering belief in relationship stability and honesty.}{This equation reveals jealousy as a complex interplay of human vulnerabilities and protective mechanisms. Possessiveness multiplies with the square of insecurity, showing how self-doubt exponentially amplifies our need to control. The logarithmic comparison term captures how we gradually accumulate resentment through social measurement. Yet security and trust form the foundation that can dissolve jealousy's poison, demonstrating that inner strength and faith in others are our greatest defenses against this consuming emotion.}